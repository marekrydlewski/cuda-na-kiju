\documentclass[12pt]{article}
\usepackage{polski}
\usepackage[utf8]{inputenc}
\usepackage{indentfirst} 
\usepackage{hyperref}
\usepackage{geometry}
\usepackage[nottoc]{tocbibind}
\newtheorem{defin}{Definicja}
 
\newcommand{\sectionbreak}{\clearpage}
 \geometry{
 a4paper,
 left=35mm,
 top=25mm,
 bottom=25mm,
 right=25mm,
 }
\begin{document}
\begin{titlepage}
	\centering
	{\scshape\LARGE Politechnika Poznańska \par}
	{\scshape\LARGE Wydział Informatyki \par}
	{\scshape\LARGE Instytut Informatyki \par}
	\vspace{1cm}
	{\scshape\Large Praca dyplomowa inżynierska\par}
	\vspace{1.5cm}
	{\huge\bfseries Implementacja algorytmu eksploracji danych z użyciem CUDA API\par}
	\vspace{2cm}
	{\Large\itshape Marcin Jabłoński \par}
	{\Large\itshape Łukasz Kosiak \par}
	{\Large\itshape Piotr Kurzawa \par}
	{\Large\itshape Marek Rydlewski \par}
	\vfill
	\begin{flushright}
	Promotor:\par
	dr inż. ~Witold \textsc{Andrzejewski}
	\end{flushright}
	\vfill
	{\large Poznań, 2017 r.\par}
\end{titlepage}
\thispagestyle{empty} % Strona z pustym stylem, bez numeru
$\mbox{ }$
\vfill\vfill
\hfill
\begin{flushright}
\begin{em}
,,Coś się popsuło`` \\
Zbigniew Stonoga
\end{em}
\end{flushright}
\vfill\pagebreak
\tableofcontents
\newpage

\section{Wstęp}

\subsection{Wprowadzenie}

Informatyzacja życia codziennego, jaka dokonała się w ostatnich latach sprawiła, że każdego dnia często nieświadomie zostawiamy po sobie wiele informacji na swój temat. Nawet z pozoru niewinne dane o naszych przyzwyczajeniach typu "z której półki bierzemy bułki w sklepie" są zapisywane w nieznanych nam systemach informatycznych. Dodając do tego inne usługi świadomie przez nas wykorzystywane - chociażby zapisywanie naszej lokalizacji przez prywatny telefon komórkowy - uzyskujemy dość ponury obraz tego, co jesteśmy w stanie po sobie zostawić. Co gorsza, chcąc czy nie chcąc, musimy się pogodzić z faktem, że dane te mogą zostać wykorzystane w różnym celu. Czy mamy jednak czego się obawiać? 

Wbrew pozorom, taka błaha na pierwszy rzut oka informacja może mieć jednak istotne znaczenie dla funkcjonowania przemysłu piekarskiego. Przecież takich informacji codziennie my, klienci, zostawiamy ogromne ilości. Nic nie szkodzi na przeszkodzie, aby spróbować z tych danych odczytać preferencje bądź przyzwyczajenia przeciętnego Kowalskiego na temat jego codziennych zakupów, które mogą w przyszłości zaprocentować - zarówno dla właściciela, jak i klienta. Jest to oczywiście tylko przykład, ale oddaje doskonale fakt przydatności z pozoru nie mających znaczenia prostych czynności człowieka, jakie często przypadkiem rejestrują działające wokół nas systemy.

Pozostaje jednak problem przetworzenia takich danych w celu otrzymania interesującej nas informacji, która byłaby potencjalnie użyteczna. Trzeba pamiętać, że rozmiar takich danych nierzadko sięga terabajtów i w praktyce skuteczna analiza takich danych przez człowieka nie jest możliwa. Musi on zatem w tym celu skorzystać z dobrodziejstw, jakie przynosi mu współczesna technologia.

Problem efektywnego przetwarzania zdążył urosnąć do rangi oddzielnego działu w informatyce. W pracy \cite{kdd} zasugerowano utworzenie nowej dyscypliny mającej na celu opracowanie technik obliczeniowych rozwiązujących takie problemy, zwanej roboczo odkrywaniem wiedzy w bazach danych (ang. KDD – \textit{Knowledge Discovery in Databases}). Techniki te mają na celu odnajdywanie prawidłowych i potencjalnie użytecznych wzorców w dużych zbiorach danych.

Wspominane wyżej techniki w dużej mierze zależą od rodzaju bazy, a ściślej mówiąc - charakteru danych występujących w niej. W przypadku danych zawierających informację o położeniu zazwyczaj mowa jest o  odkrywaniu wiedzy w bazach danych przestrzennych (ang. \textit{spatial data mining}). Takie systemy mogą zawierać atrybut lokalizacji obiektu w danym obszarze, jego opis w formie geometrycznej (np. w postaci wielokątów), a także inne atrybuty nieprzestrzenne. Okazuje się, że tradycyjne metody analizy danych przestrzennych zazwyczaj nie radzą sobie z nimi na tyle efektywnie, by było opłacalne ich użycie w praktyce \cite{trad}, dlatego też zaczęto szukać nowych sposobów na odkrywanie wiedzy w takich bazach.

W pracy \cite{huang} zaproponowano odkrywanie \textit{wzorców kolokacji przestrzennych} (lub krócej: \textit{kolokacji}), czyli zbioru cech przestrzennych występujących w niewielkiej odległości od siebie.  Łatwo to można sobie wyobrazić na przykładzie przyrody, gdzie osobniki (gatunki) o podobnych cechach zazwyczaj trzymają się razem. Rozumowanie to działa również dla bliższych współczesnemu człowiekowi cech przestrzennych, np. punktach o podobnej funkcji - stacje, kina, piekarnie, itd. Wraz z rosnącą popularnością obliczeń na kartach graficznych (w dużej mierze spowodowana wprowadzeniem technologii \textit{CUDA} autorstwa firmy NVIDIA) pojawiło się wiele gotowych rozwiązań, pozwalających na efektywne wyszukiwanie kolokacji nawet w bardzo rozbudowanych bazach danych. Przegląd niektórych z nich można znaleźć w pracy \cite{boinski}.

Ostatni rok przyniósł kolejną metodę efektywnego przeszukiwania baz danych w celu odnalezienia kolokacji \cite{chinczyki}. Wykorzystuje ona autorski algorytm wyszukiwania maksymalnych klik w grafie rzadkim oraz skondensowane drzewa instancji przechowywaczce kliki instancji dla każdego kandydata do kolokacji (patrz Rozdział 2) w celu zmniejszenia czasu obliczeń oraz ograniczenia wymagań co do pamięci operacyjnej. Algorytm ten jest przedmiotem badań niniejszej pracy zbiorowej.

\subsection{Cel i zakres pracy}

Celem niniejszej pracy jest analiza wydajności zaproponowanych w pracy \cite{chinczyki} rozwiązań z zakresu odkrywania kolokacji przestrzennych dla GPU i CPU.

Zakres pracy obejmuje następujące zadania szczegółowe:

\begin{enumerate}
\item \textbf{Zapoznanie się z literaturą.} Zapoznanie się z podstawowymi pojęciami dotyczącymi odkrywania danych w bazach danych przestrzennych oraz wyszukiwania wzorców kolokacji przestrzennych jest niezbędne do stworzenia działającej implementacji powyższego algorytmu. Dodatkowo należy zwrócić uwagę na dodatkowe zagadnienia związane z teorią grafów.
\item \textbf{Opracowanie wersji równoległej algorytmu eksploracji danych.} Konieczne jest przemyślenie wykorzystania algorytmów pomocniczych dla poszczególnych kroków całego rozwiązania oraz zaproponowanie możliwie najkorzystniejszego rozwiązania biorąc pod uwagę dostępną pamięć operacyjną, czas przetwarzania i przesyłania danych między pamięcią operacyjną a pamięcią karty graficznej.
\item \textbf{Implementacja wersji sekwencyjnej i równoległej ww. algorytmu.} Rozwiązanie podane w punkcie drugim powinno zostać zaimplementowane w technologii NVIDIA CUDA dla wersji GPU oraz biblioteki OpenMPI w przypadku odmiany dla CPU.
\item \textbf{Przeprowadzenie eksperymentów wydajnościowych.} Analiza wyników testów wydajnościowych implementacji z punktu 3 jest głównym celem tej pracy. Należy zbadać efektywność obu rozwiązań pod względem czasu wykonywania oraz zapotrzebowania na dostępną pamięć. 
\end{enumerate}

\subsection{Charakterystyka źródeł}

Jak już wspomniano, niniejsza praca w dużej mierze opiera się o algorytm zaprezentowany w dokumencie \cite{chinczyki}. Do jej opracowania była wymagana wiedza zawarta w innych źródłach, często również o charakterze naukowym.

Głównym źródłem wiedzy na temat kolokacji przestrzennych była rozprawda doktorska dr inż. Pawła Boińskiego \cite{boinski}, która w dużym przekroju omawia ideę kolokacji zaprezentowaną przez Shakara i Huanga w pracy \cite{huang}, a także prezentuje najpopularniejsze techniki ich odkrywania (metody \textit{Co-location Miner}, \textit{iCPI-tree}). Część rozwiązań wykorzystanych w tych technikach została wykorzystana w trakcie realizacji algorytmu.

Oddzielną kwestią jest literatura książkowa, wykorzystana do zapoznania się z technologią CUDA oraz przyjęcia dobrych praktyk optymalizacyjnych i programistycznych. Tutaj szczególnie należy wymienić popularną pozycję \textit{CUDA w przykładach} autorstwa Shane'a Cooke'a \cite{cuda_by_examples}, a także \textit{Professional CUDA C Programming} \cite{professional_cuda} będącą również podstawą do wstępu teoretycznego w rozdziale drugim.

\subsection{Struktura pracy}

W pracy przedstawiono główne pojęcia związane z wyszukiwaniem kolokacji przestrzennych oraz programowaniem równoległym na procesory graficzne i zawarto je w rozdziale 2. Rozdział 3 poświęcony jest algorytmowi będącemu głównym tematem pracy. Rozdział 4 opisuje implementację tego algorytmu w technologii CUDA, natomiast rozdział 5 prezentuje wyniki przeprowadzonych testów.

\textit{W tym miejscu zasadniczo będzie można napisać więcej, jeżeli już te rozdziały zostaną ustalone bądź wstępnie uzupełnione. Poza tym należy ustalić, czy w ogóle potrzebujemy takiego działu dla tak małej pracy. Z drugiej strony, zawsze to jednak te pół strony więcej spamu - borewicz}

\subsection{Podział pracy}

\textbf{Marcin Jabłoński} w ramach niniejszej pracy wykonał projekt tego i tego, opracował ......

\textbf{Łukasz Kosiak} wykonał ......, itd.

\newpage

\section{Podstawy teoretyczne}

\subsection{Charakterystyka danych przestrzennych}

\subsubsection{Modelowanie danych przestrzennych}

Sposób reprezentacji danych przestrzennej w dużej mierze zależy od zastosowań, niemniej najczęściej przybiera jedną z następujących form:

\begin{itemize}
\item \textit{model pól} - ma formę funkcji, której dziedzina należy do modelowanej przestrzeni, a jego wynikiem jest cecha przestrzenna;
\item \textit{model obiektowy} - dla każdego zjawiska jest tworzony nowy obiekt z odpowiednimi właściwościami (etykietami, atrybutami przestrzennymi i nieprzestrzennymi).
\end{itemize}

W praktyce model pól używany jest przede wszystkim w metodach opartych na dokonywaniu pomiarów z powietrza - takie dane mają wtedy charakter rastrowy (reprezentacja w postaci pikseli). Model obiektowy stosowany jest natomiast w przypadkach, gdzie występuje duża liczba dodatkowych atrybutów nieprzestrzennych.

\subsubsection{Źródła danych przestrzennych}

Najogólniej źródła danych przestrzennych można podzielić ze względu na ich format.

\textit{Pierwotne źródła danych} są opracowane w jednym ze standardowych formatów źródeł (najczęściej dla konkretnego systemu) i nie wymagają jakichkolwiek transformacji. Mają one zazwyczaj postać cyfrową i pochodzą z automatycznych pomiarów dokonanych przez specjalizowane systemy wyposażone w odbiorniki GPS czy tachimetry.

\textit{Wtórne dane źródłowe} nie zostały zebrane z myślą o wykorzystaniu w systemach typu GIS i dlatego wymagają one odpowiedniej transformacji oraz cyfryzacji (jeżeli są one analogowe). Procedury te są one obarczone pewnym ryzykiem, ponieważ istnieje możliwość wystąpienia błędów w trakcie konwersji i w konsekwencji przekłamaniami w danych wynikowych, które należy ręcznie poprawić.

\subsubsection{Relacje}

Określenie zachodzących relacji między obiektami w źródłach danych przestrzennych jest ważnym elementem przetwarzania danych przestrzennych. Sposób ich określenia zależy od zastosowanego modelu danych.

W modelu pól relacje determinowane są przez operacje pól (ang. \textit{field operations}, \cite{fieldmodel}), mogące przybierać różne formy w zależności od zastosowań, natomiast w modelu obiektowym rodzaje relacji przestrzennych zależą od definicji przestrzeni. Według standardu OGC istnieją trzy najpopularniejsze rodzaje związków przestrzennych między obiektami:

\begin{itemize}
\item \textit{Relacje metryczne} - wyrażane w postaci predykatów typu "w odległości nie większej niż 10 metrów", oparte na odległości;
\item \textit{Relacje kierunkowe} - położenie określone jest względem globalnych kierunków dla przestrzeni (np. na północ, na południe - są to relacje bezwzględne) lub względem innego obiektu/obserwatora (nazywamy takie relacjami względnymi);
\item \textit{Relacje topologiczne} - najbardziej skomplikowane, wyrażone przez zależności typu pokrywanie, zawieranie, styczność.
\end{itemize}

W systemach typu GIS stosuje się głównie relacje topologiczne. Mają one postać predykatów przestrzennych dla operacji filtrowania i połączenia przestrzennego w językach zapytań działających na danych przestrzennych. Najczęściej wykorzystuje się je w tzw. \textit{modelu dziewięciu przecięć} \cite{9sec}, za pomocą którego określa się możliwe relacje zachodzącą dla pary obiektów.

Dla każdego obiektu wyznacza się jego wnętrze, granicę i zewnętrze. Następnie, dokonuje się operacji przecięcia dla danej pary obiektów dla każdej z możliwych kombinacji elementów tego obiektu (np. granica pierwszego obiektu z wnętrzem drugiego). Takich relacji w dwuwymiarowej relacji można wyznaczyć osiem, należą do nich np. rozłączność, styczność, częściowe i całkowite pokrycie itd. 

Istnieje również rozszerzenie modelu dziewięciu przecięć, zwanym DE-9IM (ang. \textit{Dimensionally Extended nine-Intersection Model}, \cite{9sec2}), które rozróżnia rodzaj obiektu uzyskanego w wyniku przecięcia (mogą być puste, bezwymiarowe, jednowymiarowe i dwuwymiarowe). 

\subsection{Metody eksploracji danych przestrzennych}

\subsubsection{Grupowanie przestrzenne}
\subsubsection{Klasyfikacja przestrzenna}
\subsubsection{Odkrywanie trendów}
...
\subsubsection{Asocjacje przestrzenne}
\subsubsection{Kolokacje przestrzenne}

\subsection{Odkrywanie kolokacji przestrzennych}

Niniejszy rozdział zawiera opisy i definicje pojęć niezbędnych do zrozumienia algorytmu zawartego w rozdziale 3.

\subsubsection{Cecha przestrzenna}

Kluczową kwestią w procesie odkrywania kolokacji jest odpowiednia klasyfikacja obiektów występujących w bazie danych. Każdy zbiór danych przestrzennych, oprócz informacji o lokalizacji obiektu i opisujących go danych nieprzestrzennych powinien zawierać także właściwość pozwalającą na sklasyfikowanie danego obiektu do określonej klasy. Takie przypisanie nazywane jest cechą przestrzenną (ang. spatial feature) lub rzadziej klasą obiektu (ang. object class).

Jako typowy przykład cechy przestrzennej można podać etykietę przypisaną do obiektu na mapie (np. kosciół, szkoła, strzelnica). Pozwala ona na jednoznaczne określenie własności przestrzeni w punkcie, gdzie znajduje się obiekt.

\subsubsection{Podstawowe definicje}

\begin{defin}[Instancja cechy przestrzennej]
Niech f będzie cechą przestrzenną. Mówimy, że obiekt x jest instancją cechy przestrzennej f, wtedy i tylko wtedy, gdy obiekt x jest typu f oraz jest opisany przez lokalizację i identyfikator.
\end{defin}

\begin{defin}[Wzorzec i instancja kolokacji]
Załóżmy $F$ jako zbiór cech przestrzennych $F = \{ f_{1}, f_{2}, ...,f_{m} \} $ , a $FI = FI^{f_{1}} \cup FI^{f_{2}} \cup ... \cup FI^{f_{m}}$ niech będzie
zbiorem ich instancji. Niech $ >_{F} $ oznacza dowolną relację porządku zdefiniowaną dla zbioru $ F $. Niech $ f_{i} $ oznacza i-tą cechę przestrzenną (ze względu na relację $ >_{F} $), zatem $ \forall i,j \in 1,...,m $ $ f_{i} <_{F} $ $ f_{j} \Leftrightarrow i < j \land f_{i},f_{j} \in F $. Mając daną relację sąsiedztwa R (zwrotną i przechodnią) mówimy, że wzorzec kolokacji przestrzennej (w skrócie "kolokacja") jest podzbiorem cech przestrzennych $ c \subseteq F $ , których instancje $ I\subseteq FI $ tworzą klikę ze względu na relację R. Zbiór wszystkich instancji kolokacji przestrzennej $c$ jest oznaczany przez $CI^{c} $. Przez długość kolokacji należy rozumieć liczbę elementów w zbiorze cech przestrzennych, który tworzy tę kolokację.
\end{defin}

\begin{defin}[Sąsiedztwo]
Mające daną zwrotną i symetryczną relację sąsiedztwa R, sąsiedztwem lokalizacji l nazywamy zbiór lokalizacji $L = \{l_{1},l_{2}, . . . , l_{n}\}$, gdzie $l_{i}$ jest sąsiadem l, tzn. zachodzi $R(l, l_{i}) $ $ \forall i \in 1,...,n$.
\end{defin}

Przykład TODO (oprzeć na przykładzie chińczyków?)

\subsubsection{Miary kolokacji}

\begin{defin}[Współczynnik uczestnictwa]
Współczynnik uczestnictwa (ang. participation ratio) cechy f i w kolokacji c jest równy procentowemu udziałowi wszystkich instancji cechy f i w instancjach kolokacji c:

\begin{equation}
pr(f_{i}, c) = \frac{|\pi^{f_{i}}(CI^{c})|}{FI^{f_{i}}}
\end{equation}
gdzie $ \pi^{f_{i}}(CI^{c})$ oznacza projekcję relacyjną zbioru instancji $ CI^{c}$ względem cechy $f_{i}$ (z usuwaniem duplikatów).
\end{defin}

\begin{defin}[Indeks uczestnictwa]Indeks uczestnictwa (ang. participation index) kolokacji c jest równy najmniejszemu ze współczynników uczestnictwa wyznaczonych dla każdej cechy przestrzennej $ f_{i} \in c$:
\begin{equation}
pi(c) = min_{f_{i} \in c} pr(f_{i} ,c)
\end{equation}
Indeks uczestnictwa najczęściej określany jest w literaturze mianem miary powszechności lub krótko powszechnością kolokacji.
\end{defin}

\begin{defin}[Maksymalny wzorzec kolokacji przestrzennej]Niech będzie dana wartość min\_prev oznaczająca pewien minimalny próg powszechności. Jeżeli $ c = \{f_{1},...,f_{m} \} $ jest kolokacją powszechną (tzn. $ pi(c) \ge min\_prev $) i nie istnieje żaden nadzbiór c taki, że powszechność dla tego nadzbioru jest równa co najmniej min\_prev, kolokacja c nazywana jest kolokacją maksymalną.  
\end{defin}

\subsubsection{Problem}

\begin{defin}[Reguła kolokacyjna]Reguła kolokacyjna to reguła postaci $ c_{1} \rightarrow c_{2}(p, cp)$, gdzie $ c_{1} \subseteq F $, $c_{2} \subseteq F $ i $c _{1} \cup c_{2} = \emptyset $. Potencjalna użyteczność reguły może być mierzona przy pomocy jej powszechności p oraz prawdopodobieństwa warunkowego cp.
\end{defin}

\begin{defin}[Prawdopodobieństwo warunkowe]Prawdopodobieństwem warunkowym
$ cp(c_{1}, c_{2})$ reguły kolokacyjnej $ c_{1} \rightarrow c_{2} $ nazywamy stosunek liczby instancji wzorca c 1 w sąsiedztwie instancji wzorca $ c_{2}$ do liczby wszystkich instancji wzorca $ c_{1} $:
\begin{equation}
cp(c_{1}, c_{2}) = \frac{|\pi^{c_{1}}(CI^{c_{1} \cup c_{2}})|}{CI^{c_{1}}}
\end{equation}
gdzie $\pi^{c_{1}}(CI^{c_{1} \cup c_{2}})$ oznacza projekcję relacyjną instancji wzorca $CI^{c_{1} \cup c_{2}}$ względem wzorca $ c_{1} $ (z usuwaniem duplikatów).
\end{defin}

\begin{defin}[Problem odkrywania kolokacji]
Problem odkrywania kolokacji przestrzennych jest zdefiniowany w następujący sposób.
Mając dane:
\begin{itemize}
\item zbiór cech przestrzennych $F = \{ f_{1}, f_{2}, ...,f_{m} \} $
\item zbiór obiektów $FI = FI^{f_{1}} \cup FI^{f_{2}} \cup ... \cup FI^{f_{m}}$  , gdzie $ FI^{f_{i}},(0 < i \le m) $ jest zbiorem instancji cechy $ f_{i}$, przy czym każda instancja jest opisana przez lokalizację i identyfikator,
\item symetryczną i zwrotną relację sąsiedztwa $R$,
\item próg minimalnej powszechności min\_prev oraz próg minimalnego prawdopodobieństwa warunkowego min\_cond,
\end{itemize}
znajdź wszystkie poprawne reguły kolokacyjne z powszechnością nie mniejszą niż min\_prev i prawdopodobieństwem warunkowym nie mniejszym niż min\_cond.
\end{defin}

\subsection{Przegląd algorytmów odkrywania wzorców kolokacji przestrzennych}

W tym podrozdziale zostaną zaprezentowane skrótowo najważniejsze algorytmy odkrywania kolokacji przestrzennych. 

\subsubsection{Co-location Miner}

Wraz z wprowadzeniem pojęcia kolokacji autorzy pracy \cite{huang} zaprezentowali także podstawowy obecnie algorytm rozwiązujący problem odkrywania wzorców kolokacji przestrzennych, zwany \textit{Co-location Miner}. W algorytmie tym wyróżnia się następujące fazy:

\begin{itemize}
\item generowanie kandydatów na kolokacje przestrzenne (o długości $i$),
\item wyznaczanie instancji dla wygenerowanych kandydatów,
\item usuwanie kandydatów, których powszechność wynosi mniej niż przyjęty próg minimalnej powszechności.
\end{itemize}

Pozostali kandydaci trafiają do zbioru wynikowego, a następnie na ich podstawie są tworzone reguły kolokacyjne. Same reguły również podlegają filtracji - usuwane są te reguły, których prawdopodobieństwo warunkowe jest poniżej określonego progu. 

W następnej iteracji algorytm wykonuje dokładnie te same kroki, przy czym generowani kandydaci są o długości o jeden większej. Całość kończy się, gdy nie jest możliwe już wygenerowanie nowych kandydatów.

\subsubsection{Multiresolution Co-location Miner}

Korzystanie z oryginalnego algorytmu \textit{Co-location Miner} wiąże się niestety z dużymi kosztami obliczeniowymi, głównie ze względu na pracochłonny krok generowania kandydatów na kolokacje. Dlatego też niedługo później w pracy \cite{multihuang} autorzy zaproponowali drobną modyfikację oryginalnego algorytmu, dodając dodatkowy krok filtrowania w opraciu o przybliżoną reprezentację zbioru wejściowego.

W algorytmie \textit{Multiresolution Co-location Miner} zbiór wejściowy zostaje podzielony na obszary (mniejsze fragmenty). Zanim rozpocznie się faza wyznaczania instancji dla wygenerowanych kandydatów, nastepuje szacowanie ich powszechności na podstawie sąsiadujących instancji cech przestrzennych w ramach obszarów. W przypadku zbyt niskiej wartości szacowanej powszechności kandydata, można go wykluczyć z dalszego przetwarzania i tym samym oszczędzić zasoby niezbędne na wyznaczenie jego instancji.

Dalsze kroki przebiegają identycznie jak w przypadku oryginalnego \textit{Co-location Miner}.

\subsubsection{Joinless}

Celem autorów pracy \cite{joinless} było stworzenie algorytmu, który omijałby konieczność tworzenia kosztownych połączeń przestrzennych na etapie wyznaczania instancji kandydatów na kolokacje (tak jak np. w przypadku rodziny algorytmów \textit{Co-location Miner}). Nosi on nazwę algorytmu bezpołączeniowego (ang. \textit{joinless}).

Główną różnicą w porównaniu do wcześniejszych algorytmów jest sposób generowania instancji kolokacji. Są one generowane na podstawie sąsiedztw typu gwiazda - zbiorów obiektów, w którego skład wchodzi rozpatrywany obiekt oraz jego sąsiedzi posiadający większą cechę przestrzenną. Wyznacza się je na podstawie oddzielnych algorytmów (np. \textit{plane sweep}), lub korzysta z specjalnych struktur ułatwiających wykrywanie sąsiadów typu \textit{R-drzewo}.

Wygenerowane instancje muszą zostać dodatkowo zweryfikowane (poprawne instancje powinny być kliką, czego nie gwarantuje sąsiedztwo typu gwiazda), a następnie - podobnie jak w algorytmie \textit{Multiresolution Co-location Miner} - dokonuje się ich wstępnego filtrowania pod kątem progu minimalnej powszechności.

\subsubsection{iCPI-tree}

Drzewo iCPI (\textit{improved Co-location Pattern Instance}, \cite{icpi}) stanowi zmodyfikowaną odmianę drzewa CPI zawartego w pracy \cite{cpi}. Struktura ta zawiera informacje o wszystkich zachodzących relacjach sąsiedztwa. 

\textit{iCPI-tree} posiada następującą strukturę:

\begin{itemize}
\item Poziom 1 - korzeń drzewa (oznaczony etykietą \textit{NULL}),
\item Poziom 2 - cechy elementów centralnych, czyli cechy przestrzenne obiektów centralnych \textit{sąsiedztw typu gwiazda};
\item Poziom 3 - instancje elementów centralnych, dla których ma zostać przechowana informacja o sąsiadach;
\item Poziom 4 - cechy sąsiadów,
\item Poziom 5 - instancje sąsiadów.
\end{itemize}

Sąsiedzi uporządkowani są według rosnącej cechy przestrzennej, a w przypadku instancji tej samej cechy - zgodnie z rosnącym identyfikatorem. Takie uporządkowanie nosi nazwę \textit{uporządkowanego zbioru sąsiadów}.

Powyższa struktura drzewiasta jest wykorzystana w algorytmie w celu generowania instancji coraz dłuższych kandydatów w kolejnych iteracjach. Dokonuje się tego poprzez systematyczną ich rozbudowę o kolejne elementy. Na początku wszystkie instancje są jednoelementowe, a w kolejnych iteracjach są one rozbudowywane poprzez wyszukiwanie sąsiadów z odpowiednią cechą i weryfikowane (należy sprawdzić, czy nowo dodany obiekt do instancji jest sąsiadem każdego z obiektów należących do tej instancji). 

Pozostałe kroki algorytmu (generowanie kandydatów i reguł, filtrowanie według powszechności) są podobne jak w metodach \textit{Co-location Miner} i \textit{joinless}.

\newpage

\section{Algorytm}

\subsection{Konstrukcja tabeli instancji o rozmiarze 2}

\subsection{Obliczanie miary powszechności}

\subsection{Generowanie kandydatów na kolokacje maksymalne}

\subsection{Proces odcinania}


\section{Implementacja}

\section{Testy efektywnościowe}

\section{Zakończenie}

\newpage

\begin{thebibliography}{}
\bibitem{kdd}Usama Fayyad, Gregory Piatetsky-Shapiro, and Padhraic Smyth. From Data Mining
to Knowledge Discovery in Databases. AI Magazine, 17:37–54, 1996.
\bibitem{sop}Ł.~Stanisławowski. \emph{Bogactwo i nędza narodów.}
O'reilly, 2013.
\bibitem{trad} Harvey J. Miller and Jiawei Han. Geographic Data Mining and Knowledge Discovery.
Taylor \& Francis, Inc., Bristol, PA, USA, 2001
\bibitem{huang} S. Shekhar and Y. Huang. Discovering Spatial Co-location Patterns: A Summary of Results. In SSTD 2001, pages 236–256, 2001.
\bibitem{boinski} Przetwarzanie zbiorów przestrzennych zapytan neksploracyjnych w srodowiskachzograniczonym rozmiarem pamiecioperacyjnej
\bibitem{chinczyki}A fast space-saving algorithm for maximal co-location pattern mining
\bibitem{cuda_by_examples}CUDA by Example: An Introduction to General-Purpose GPU Programming, Jason Sanders, Edward Kandrot
\bibitem{professional_cuda}Professional CUDA C Programming, John Cheng, Max Grossman, Ty McKerche
\bibitem{multihuang}Shashi Shekhar and Yan Huang. The Multi-resolution Co-location Miner: A New Algorithm to Find Co-location Patterns in Spatial Dataset. Technical Report 02-019, University of Minnesota, 2002.
\bibitem{joinless}Jin Soung Yoo and Shashi Shekhar. A Joinless Approach for Mining Spatial Colocation Patterns. IEEE Transactions on Knowledge and Data Engineering, 18(10):13231337, 2006.
\bibitem{cpi}Lizhen Wang, Yuzhen Bao, Joan Lu, and Jim Yip. A New Join-less Approach for Co-location Pattern Mining. In Qiang Wu, Xiangjian He, Quang Vinh Nguyen, Wenjing Jia, and Mao Lin Huang, editors, Proceedings of the 8th IEEE International
Conference on Computer and Information Technology (CIT 2008), pages 197–202, Sydney, July 2008. IEEE.
\bibitem{icpi}Lizhen Wang, Yuzhen Bao, and Joan Lu. Efficient Discovery of Spatial Co-Location Patterns Using the iCPI-tree. The Open Information Systems Journal, 3(2):69–80,2009.
\bibitem{fieldmodel}Christopher Jones and Mark Hall. A Field Based Representation for Vague Areas Defined by Spatial Prepositions. In Proceedings of the Workshop on Methodologies and Resources for Processing Spatial Language at 6th Language Resources and Evaluation Conference (LREC 2008), 2008.
\bibitem{9sec}Max J. Egenhofer and Robert Franzosa. Point-set topological spatial relations. International Journal of Geographic Information Systems, 5(2):161–174, 1991.
\bibitem{9sec2} Eliseo Clementini, Paolino Di Felice, and Peter van Oosterom. A small set of formal
topological relationships suitable for end-user interaction. In Proceedings of the 3rd
International Symposium on Advances in Spatial Databases (SSD 1993), pages 277–
295, London, UK, UK, 1993. Springer-Verlag.
\end{thebibliography}

\newpage

\appendix

\section{Dodatek A}

\section{Dodatek B}

\end{document}