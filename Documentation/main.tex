\documentclass{article}
\usepackage{polski}
\usepackage[utf8]{inputenc}
\usepackage{indentfirst} 
\usepackage{hyperref}
\usepackage{geometry}
\usepackage[nottoc]{tocbibind}
\newcommand{\sectionbreak}{\clearpage}
 \geometry{
 a4paper,
 left=35mm,
 top=25mm,
 bottom=25mm,
 right=25mm,
 }
\begin{document}
\begin{titlepage}
	\centering
	{\scshape\LARGE Politechnika Poznańska \par}
	{\scshape\LARGE Wydział Informatyki \par}
	{\scshape\LARGE Instytut Informatyki \par}
	\vspace{1cm}
	{\scshape\Large Praca dyplomowa inżynierska\par}
	\vspace{1.5cm}
	{\huge\bfseries Implementacja algorytmu eksploracji danych z użyciem CUDA API\par}
	\vspace{2cm}
	{\Large\itshape Marcin Jabłoński \par}
	{\Large\itshape Łukasz Kosiak \par}
	{\Large\itshape Piotr Kurzawa \par}
	{\Large\itshape Marek Rydlewski \par}
	\vfill
	\begin{flushright}
	Promotor:\par
	dr inż. ~Witold \textsc{Andrzejewski}
	\end{flushright}
	\vfill
	{\large Poznań, 2017 r.\par}
\end{titlepage}
\thispagestyle{empty} % Strona z pustym stylem, bez numeru
$\mbox{ }$
\vfill\vfill
\hfill
\begin{flushright}
\begin{em}
Nikomu nie dziękujemy. \\
\end{em}
\end{flushright}
\vfill\pagebreak
\tableofcontents

\section{Wstęp}

Wprowadzenie do tematu...

\subsection{Cel i zakres pracy}
Celem niniejszej pracy jest...

\section{Podstawy teoretyczne}

Więcej informacji można znaleźć w książce \cite{sop}.

\section{Algorytm}

\section{Implementacja}

\section{Testy efektywnościowe}

\section{Zakończenie}

\begin{thebibliography}{}
\bibitem{sop}Ł.~Stanisławowski. \emph{Bogactwo i nędza narodów.}
O'reilly, 2013.
\end{thebibliography}

\appendix

\section{Dodatek A}

\section{Dodatek B}

\end{document}
